\documentclass[12pt]{article}
\usepackage{myart}\addbibresource{/Users/lwg342/Documents/LaTeX/lib.bib}
%--------Hello World--------%
%=====================================%
\begin{document}
    \title{Prediction with Interval Outcomes}

    \section{Procedure Summary}

    \blue{Blue items are parameters of simulation designs that can be changed.}

    We randomly generate \blue{sample size \(N\)} observations of \((y_{i},x_{i}), i = 1 ,\dots, N\), where \(x\) is \blue{\(K\)-dimensional}, and
    \begin{equation*}
        y_{i} = \blue{f}(x_{i}) + \epsilon_{i}.
    \end{equation*}
    where \(\epsilon\) are independent draws from from either a normal distribution with variance \blue{\(v_{\epsilon}\)} or a centered \(\chi^{2}\) distribution with \blue{\(\text{df}\)} degrees of freedom \(\epsilon_{i} \sim \frac{\chi^{2}_{\text{df}} - \text{df}}{\sqrt{2 \text{df}}}\). 
    Let's assume we observe not \(y_{i}\), but \(y^{l}_{i}\) and \(y_{i}^{u}\). In the simulation, it's generated by 
    \begin{align*}
        y_{i}^{l} &= \lfloor y_{i}/b \rfloor \times b - a_{1} \\
        y_{i}^{u} &= \lceil y_{i}/b \rceil \times b  + a_{2},
    \end{align*}
    where \blue{b} will control the scale/length of the interval observations and \blue{\(a_{1},a_{2}\)} will control the bias, in what follows let's first take \(a_{1} = a_{2} = 0\). Let all \(\beta\)'s be \(1\), and \(x_{i}\) either have \blue{normal \(N(0,I_{k})\) or uniform distribution \(U( - \sqrt{3},\sqrt{3})\)}. We follow Elie's suggestions and do the following:
    \begin{enumerate}
        \item Split the sample into training and testing sets. 
        \item Randomly pick \(\lambda^{m} \sim \blue{\text{Beta}(p_{1}, p_{2})}\),  \(y^{m}_{i} = \lambda^{m}y^{l}_{i} + (1 - \lambda^{m}) y^{u}_{i}\), \(m = 1,\dots, \blue{M}\), and use them as the training set. Fit \(M\) \blue{regression models} \(\hat{f}_{m}\) of choice. Let \(\tilde{Y}(x) = \left\{\hat{f}_{m}(x): 1\leq m\leq M\right\}\) be the \textit{pre-selection prediction sets} for a given evaluation point \(x\), which is the set of predictions from all models.
        \item Compute the score \(S_{m}\) for each model on the testing set, find the  minimum \(S_{m}^{*}\) and \(\mathcal{M}^{*} = \Bqty{m: S_{m} \leq S_{m}^{*} + \tau}\), where \blue{\(\tau = \frac{1}{\sqrt{n}}\)} is the \blue{tolerance}. And the \textit{post-selection prediction sets} are \(\hat{Y}(x) = \Bqty{\hat{f}_{m}(x): m\in \mathcal{M}^{*}}\).
    \end{enumerate}

    \section{Simulations about the random draws \tmath{y_{d}}}

    There are two considerations about the ways we draw \(y_{d}\). Firstly, we need to decide on the number of draws. Secondly, we need to decide on the distribution of the draws. 
% The End
% \printbibliography
\end{document}